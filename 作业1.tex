\documentclass[UTF8,12pt]{ctexart}

\usepackage[a4paper,margin=2.2cm]{geometry}
\usepackage{amsmath,amssymb,bm}
\usepackage{hyperref}
\hypersetup{colorlinks=true,linkcolor=blue,urlcolor=blue,citecolor=blue}

\newcommand{\j}{\mathrm{j}}
\newcommand{\e}{\mathrm{e}}

\title{矩形波导中的电磁波传播求解(期中作业)\\TE/TM 模式、截止频率与群速度/相速度}
\author{姓名:陈关棠\quad 学号:2024210642}}}
\date{\today}

\begin{document}
\maketitle

\section{引言与模型}
考虑理想金属(PEC)矩形波导,截面为 $0<x<a,\,0<y<b$,沿 $z$ 方向均匀延伸,内部充满均匀介质 $(\varepsilon,\mu)$。采用时间谐波 $\e^{\j\omega t}$,并设沿 $z$ 的传播形式为 $\e^{-\j\beta z}$。

无源频域麦克斯韦方程:
\begin{align}
\nabla\times \bm{E} &= -\j\omega\mu\,\bm{H},\qquad
\nabla\times \bm{H} = \j\omega\varepsilon\,\bm{E},\\
\nabla\cdot \bm{E} &= 0,\qquad
\nabla\cdot \bm{H} = 0,
\end{align}
定义介质波数 $k=\omega\sqrt{\mu\varepsilon}$。

\paragraph{边界条件(PEC)}
金属壁面上电场切向分量为零:$\bm{E}_t=0$。

\section{TE/TM 模式与求解思路}
把场分成横向与纵向:$\bm{E}=\bm{E}_t+\hat{\bm{z}}E_z,\ \bm{H}=\bm{H}_t+\hat{\bm{z}}H_z$。
常见导波模式分两类:
\[
\text{TE: }E_z=0,\ H_z\neq 0;\qquad
\text{TM: }H_z=0,\ E_z\neq 0.
\]
波导问题的常用做法是:先求纵向分量(TE 求 $H_z$,TM 求 $E_z$),再由麦克斯韦方程得到其余分量。

\paragraph{纵向分量满足的二维方程(教材常用结论)}
在均匀波导中可得到
\begin{equation}
\left(\frac{\partial^2}{\partial x^2}+\frac{\partial^2}{\partial y^2}+k_c^2\right)\{H_z \text{ 或 } E_z\}=0,
\qquad
\boxed{\ \beta^2=k^2-k_c^2\ },
\end{equation}
其中 $k_c$ 为截止波数。

\section{TE 模式($E_z=0$)}
\subsection{边界条件转化}
TE 模下 $\bm{E}_t$ 由 $H_z$ 的横向变化产生;结合 $\bm{E}_t=0$ 可得到
\[
\boxed{\ \frac{\partial H_z}{\partial n}=0\ \ \text{在四壁上}\ }
\]
(即 $H_z$ 满足 Neumann 边界条件)。

\subsection{分离变量解}
设 $H_z=X(x)Y(y)$,结合边界条件可得到余弦型解:
\begin{equation}
\boxed{
H_z=H_0\cos\left(\frac{m\pi x}{a}\right)\cos\left(\frac{n\pi y}{b}\right)\e^{-\j\beta z}
}
\end{equation}
其中 $m,n=0,1,2,\dots$ 且 $(m,n)\neq(0,0)$。

对应截止波数:
\begin{equation}
\boxed{
k_c^2=\left(\frac{m\pi}{a}\right)^2+\left(\frac{n\pi}{b}\right)^2
}
\end{equation}

\subsection{其余场分量(直接套用的表达式)}
\[
\boxed{
\begin{aligned}
E_x &= \frac{\j\omega\mu}{k_c^2}\frac{\partial H_z}{\partial y},\quad
E_y = -\frac{\j\omega\mu}{k_c^2}\frac{\partial H_z}{\partial x},\quad E_z=0,\\
H_x &= -\frac{\j\beta}{k_c^2}\frac{\partial H_z}{\partial x},\quad
H_y = -\frac{\j\beta}{k_c^2}\frac{\partial H_z}{\partial y}.
\end{aligned}}
\]

\section{TM 模式($H_z=0$)}
\subsection{边界条件转化}
TM 模中由 PEC 条件可推出
\[
\boxed{\ E_z=0\ \ \text{在四壁上}\ }
\]
(Dirichlet 边界条件)。

\subsection{分离变量解}
因此 $E_z$ 为正弦型:
\begin{equation}
\boxed{
E_z=E_0\sin\left(\frac{m\pi x}{a}\right)\sin\left(\frac{n\pi y}{b}\right)\e^{-\j\beta z}
}
\end{equation}
其中 $m,n=1,2,3,\dots$。

截止波数同样为
\[
\boxed{
k_c^2=\left(\frac{m\pi}{a}\right)^2+\left(\frac{n\pi}{b}\right)^2
}.
\]

\subsection{其余场分量}
\[
\boxed{
\begin{aligned}
E_x &= -\frac{\j\beta}{k_c^2}\frac{\partial E_z}{\partial x},\quad
E_y = -\frac{\j\beta}{k_c^2}\frac{\partial E_z}{\partial y},\quad H_z=0,\\
H_x &= -\frac{\j\omega\varepsilon}{k_c^2}\frac{\partial E_z}{\partial y},\quad
H_y = \frac{\j\omega\varepsilon}{k_c^2}\frac{\partial E_z}{\partial x}.
\end{aligned}}
\]

\section{验证:电场与磁场不能同时为横波(无 TEM 模式)}
TEM 的定义是 $E_z=0$ 且 $H_z=0$。下面说明在空心单导体矩形波导中 TEM 只能得到零场。

若 $H_z=0$,由 $\nabla\times\bm{E}=-\j\omega\mu\bm{H}$ 取 $z$ 分量得
\[
\frac{\partial E_y}{\partial x}-\frac{\partial E_x}{\partial y}=0
\Rightarrow \bm{E}_t=-\nabla_t\phi.
\]
再由 $\nabla\cdot\bm{E}=0$ 且 $E_z=0$ 得
\[
\nabla_t\cdot\bm{E}_t=0 \Rightarrow \nabla_t^2\phi=0.
\]
PEC 边界上 $\bm{E}_t=0$,等价于 $\nabla_t\phi=0$,从而边界上 $\phi$ 为常数。对矩形这种单连通区域,满足拉普拉斯方程且边界为常数的解只能是全域常数,因此 $\bm{E}_t=0$;再代回麦克斯韦方程可知 $\bm{H}_t=0$,最终只有零场。
\[
\boxed{\ \text{矩形单导体波导不存在非零 TEM 模式,因此 }\bm{E},\bm{H}\text{不能同时为横波}\ }.
\]

\section{截止频率}
由 $\beta^2=k^2-k_c^2$ 可知:当 $k<k_c$ 时 $\beta$ 变为虚数,沿 $z$ 方向不传播而衰减,因此截止条件为 $k=k_c$。
\[
\omega_c=\frac{k_c}{\sqrt{\mu\varepsilon}},\qquad f_c=\frac{\omega_c}{2\pi}.
\]
代入 $k_c$:
\begin{equation}
\boxed{
f_c=\frac{1}{2\sqrt{\mu\varepsilon}}
\sqrt{\left(\frac{m}{a}\right)^2+\left(\frac{n}{b}\right)^2}
}
\end{equation}
真空中 $1/\sqrt{\mu\varepsilon}=c$,则
\[
\boxed{
f_c=\frac{c}{2}\sqrt{\left(\frac{m}{a}\right)^2+\left(\frac{n}{b}\right)^2}.
}
\]

\section{群速度与相速度关系(相速度可大于光速)}
色散关系:
\[
\beta(\omega)=\sqrt{\omega^2\mu\varepsilon-k_c^2},\qquad (\omega>\omega_c).
\]
相速度 $v_p=\omega/\beta$:
\[
v_p=\frac{1}{\sqrt{\mu\varepsilon}}\cdot\frac{1}{\sqrt{1-(\omega_c/\omega)^2}}
\Rightarrow \boxed{v_p> \frac{1}{\sqrt{\mu\varepsilon}} }.
\]
群速度 $v_g=\dd\omega/\dd\beta$,由求导可得
\[
v_g=\frac{1}{\sqrt{\mu\varepsilon}}\sqrt{1-(\omega_c/\omega)^2}
\Rightarrow \boxed{v_g< \frac{1}{\sqrt{\mu\varepsilon}} }.
\]
并且两者满足一个很好记的结论:
\[
\boxed{\,v_p v_g=\frac{1}{\mu\varepsilon}\,}.
\]
在真空中 $1/\sqrt{\mu\varepsilon}=c$,因此 $v_p>c$ 而 $v_g<c$。相速度大于光速不代表信息超光速传播,通常与能量/信息相关的是群速度。

\section{结论}
本文从麦克斯韦方程出发,给出了矩形波导中 TE/TM 两类模式的纵向分量形式与其余分量的通用表达式;证明了空心矩形单导体波导不存在 TEM 模式;推导了模式截止频率;并给出群速度、相速度的关系 $v_p v_g=1/(\mu\varepsilon)$,说明相速度可以大于光速而群速度小于光速。

\end{document}
